\documentclass[a4paper, 12pt, twoside]{book}
\usepackage{fontspec}
\usepackage{babel}
\usepackage[utf8x]{inputenc}
\usepackage[T1]{fontenc}
\usepackage{fontspec}
\usepackage{lmodern}
\usepackage{graphicx}
\usepackage{enumitem}
\usepackage{lscape}
\usepackage{subcaption}
\usepackage{imakeidx}
\usepackage{tocbibind}
\usepackage{hyperref}

\usepackage{listings}
\lstset{%
	basicstyle=\footnotesize\ttfamily,%
	numbers=left,%
	backgroundcolor=\color{lightgray},%
	breaklines=true%
}
\usepackage[newfloat]{minted}
\SetupFloatingEnvironment{listing}{}
\usemintedstyle{emacs}
\setminted{linenos, breaklines, tabsize=4, bgcolor=lightgray}

\usepackage[margin=2.5cm]{geometry}
\usepackage{setspace}
\setlength\parindent{1cm}
\onehalfspacing

\begin{document}

\chapter{\texttt{md2tex} - a CLI to convert Markdown to TeX}

\texttt{md2tex} is a simple but customizable markdown to TeX converter inspired by XSLT and functionnal programming. 
Rather than using fancy object oriented libraries, it locates elements of Markdown syntax using regular 
expressions and translates them into TeX syntax. It has only been tested with latin script so far.

\par\noindent\rule{\linewidth}{0.4pt}
\section{Features}

\begin{itemize}
\item Translation from Markdown to TeX
\item Creation of a complete and valid TeX document, either from a generic template or using your own template
\item Reading Markdown from custom locations and writing a TeX file to a custom location
\item Several options for extra customization: type of quotes inline quotes, numbered or unnumbered headers, document classes... 
\end{itemize}

\par\noindent\rule{\linewidth}{0.4pt}
\section{Commands / How to}

The advantage of having a script focused on Markdown to TeX conversion is that you can get a lot of fine tuning
and extra customization.
\subsection{Requirements}

The script relies on the \href{http://tug.ctan.org/macros/latex/contrib/minted/minted.pdf}{\texttt{minted}}\footnote{Website visited on 02.08.2022.} package,
so this needs to be installed on your machine. This script has been tested on a Linux machine and should
work on UNIX and affiliated systems. Theoritically, it should also run on Windows, although that has not
been tested. You will need to tailor the below scripts if you use Windows.
\subsection{Automated installation (MacOS/GNU-Linux)}

An installation shell script has been written to make things easier.
Simply clone the repo and launch the install !

\begin{listing}[h!]
   \begin{minted}{bash}
git clone https://github.com/paulhectork/md2tex.git # clone the repo
cd md2tex # move to the directory
bash install.sh # can also be used with other shells: zsh, ash...

   \end{minted}
\end{listing}
\subsection{Manual install}

If the above method doesn't work, a manual installation is just as easy:

\begin{listing}[h!]
   \begin{minted}{bash}
git clone https://github.com/paulhectork/md2tex.git # clone the repo
cd md2tex # move to the directory
python3 -m venv env # create virtual env
source env/bin/activate # source it
pip install -r requirements.txt # install requirements
pip install --editable . # build the package

   \end{minted}
\end{listing}
\subsection{Base functionning}

This CLI should be used from the \texttt{md2tex} directory.

\begin{listing}[h!]
   \begin{minted}{bash}
source env/bin/activate # get the proper env
md2tex path/to/markdown/file.md --options

   \end{minted}
\end{listing}
\subsection{Arguments and options}
\subsubsection{Argument}

The only \textbf{compulsory argument} is the path to the markdown file that needs to be processed.

\begin{itemize}
\item The file must finish with \texttt{.md} so that we're sure that a markdown file is being processed. 
\end{itemize}
\subsubsection{Optional parameters}

As you can see below, there quite a few possibilities for fine-tuning. All of the below
parameters are optional.

\begin{itemize}
\item \textbf{\texttt{o}, \texttt{-output-path}}: give a specific path and filename to write the \texttt{.tex} file to?
\begin{itemize} 
 \item if none is provided, the output path defaults to \texttt{md2tex/output/input\_filename.tex}
\item if the output path contain directories that do not exist, the output will not be 	 written. Directories must be created manually beforehand.
\begin{itemize} 
 \item if the extension of the output file is not \texttt{.tex}, a \texttt{.tex} extension will be added to the filename.
\end{itemize}
\end{itemize}
\item \textbf{\texttt{-c}, \texttt{--complete-tex-file}}: if provided, a complete TeX file will be created, with preamble and table of contents. 
\begin{itemize} 
 \item the default template used is \texttt{utils/template.tex}. For a custom template, see \texttt{-t} below.
\item if not provided, only the body of the Markdown file will be converted. It will need to be included 	 in another file with a premable.
\end{itemize}
\item \textbf{\texttt{-t}, \texttt{--custom-tex-template}}: the path to a custom TeX template to build a complete document from, instead of the default template \texttt{utils/template.tex}.
\begin{itemize} 
 \item this argument must be used in conjunction with \texttt{-c} for an actual template to be used.
\item a path to an existing template must be provided.
\item this template must contain the string \texttt{@@BODYTOKEN@@}: this token will be replaced by the contents of the 	 converted Markdown file.
\item the document class can also be defined in the document by the CLI: 	 in the template's preamble, write \texttt{documentclass\{book\}}.
\end{itemize}
\item \textbf{\texttt{-d}, \texttt{--document-class}}: this argument indicates the class of the final TeX document. it indicates wether to process the Markdown file as a LaTeX \texttt{book} or as an \texttt{article}.
\begin{itemize} 
 \item defaults to \texttt{article}.
\item if used with a custom TeX template (see \texttt{-c} and \texttt{-t}), the template must contain an \texttt{book} 	 in the preamble where the document class is specified so that the TeX document class can actually change.
\item in fact, this argument only impacts the headers used in the TeX template.
\end{itemize}
\item \textbf{\texttt{-u}, \texttt{--unnumbered-headers}}: if this argument is provided, the TeX headers (\texttt{\textbackslash{}chatper}, \texttt{section}...) will be unnumbered.
\begin{itemize} 
 \item by default, the headers are numbered.
\end{itemize}
\item \textbf{\texttt{-f}, \texttt{--french-quote}}: if this argument is provided, anglo-saxon inline quotes will be replaced by french quotes using the \texttt{\textbackslash{}enquote} command.
\begin{itemize} 
 \item defaults to False: anglo-saxon quotes are used. 
\end{itemize}
\end{itemize}
\subsection{Command line help}

\begin{listing}[h!]
   \begin{minted}{bash}
md2tex --help

   \end{minted}
\end{listing}

\par\noindent\rule{\linewidth}{0.4pt}
\section{Examples}

Output TeX files and the PDFs (using \texttt{XeLaTeX}) can be seen in the \texttt{readme/} and \texttt{examples/} directory.

\begin{itemize}
\item the \texttt{readme/} directory contains the canonical TeX and PDF documentation, produced by processing this README file.
\item the \texttt{examples/} directory contains additional examples, including different parameters used to process this file.
\item the script \texttt{readme\_conversion.sh} allows to run a series of conversions and compilation of this file into different TeX and PDF files. 
\end{itemize}

\textbf{Example 1} - the canonical \texttt{README.tex} file in the \texttt{README} dir:

\begin{itemize}
\item a complete document is created (\texttt{-c})
\item it uses the custom template (\texttt{-t}) \texttt{./readme/readme\_article\_template.tex}
\item the output is written (\texttt{-o}) to \texttt{./readme/README.tex}
\item the document created is an article (\texttt{-d article})
\item the document uses unnumbered headers (\texttt{-u}) and french quotes (\texttt{-f}) 
\end{itemize}

\begin{listing}[h!]
   \begin{minted}{bash}
md2tex README.md -c -t ./readme/readme_article_template.tex -o ./readme/README.tex -d article -u -f

   \end{minted}
\end{listing}

\textbf{Example 2} - a partial TeX file to a custom output (\texttt{-o ./examples/README\_partial\_base.tex}) with default params

\begin{listing}[h!]
   \begin{minted}{bash}
md2tex README.md -o ./examples/README_partial_base.tex

   \end{minted}
\end{listing}

\textbf{Example 3} - a complete document of class \texttt{book} using a custom template with unnumbered headers and french quotes

\begin{listing}[h!]
   \begin{minted}{bash}
md2tex README.md -c -t ./readme/readme_book_template.tex -o ./examples/README_book.tex -d book -u -f

   \end{minted}
\end{listing}

\textbf{Example 4} - a complete document of class \texttt{article} with a custom template and numbered (default) headers

\begin{listing}[h!]
   \begin{minted}{bash}
md2tex README.md -c -t ./readme/readme_article_template.tex -o ./examples/README_article_numbered.tex

   \end{minted}
\end{listing}

\textbf{Example 5} - a complete document of class \texttt{book} with default template and numbered headers

\begin{listing}[h!]
   \begin{minted}{bash}
md2tex README.md -c -o ./examples/README_article_default_template_numbered.tex

   \end{minted}
\end{listing}

\textbf{Example 6} - a complete document of class \texttt{article} with default template and default params

\begin{listing}[h!]
   \begin{minted}{bash}
md2tex README.md -c -o ./examples/README_book_default_template.tex -d book

   \end{minted}
\end{listing}

\par\noindent\rule{\linewidth}{0.4pt}
\section{Markdown syntax}
\subsection{Currently supported}

\begin{itemize}
\item Line breaks: paragraph changes using \texttt{\textbackslash{}n\textbackslash{}n}, \texttt{<br\textgreater{}} and \texttt{<br/\textgreater{}}
\item Different styles of text: \textbf{bold}, \textit{italic} and \texttt{inline code}. \textbf{Warning}:
\begin{itemize} 
 \item only bold and italics made using asterisks will be translated.
\end{itemize}
\item Block quotes (lines beginning with \texttt{\textgreater{}}). \textbf{Warning}: only non-nested block quotes -- or the outer level of a nested block quote -- will be rendered.
\item Multiline code; if possible, the code is colored using \texttt{minted}. Indentation levels are \textbf{always} respected within multiline code.
\item Ordrered and unordered lists, including nested lists. \textbf{Warning}:
\begin{itemize} 
 \item To be processed, all indentation levels must be a multiplier of the indentation of the first indented item. See below for details on how nested lists are handled.
\item Unordered lists will only be converted if they begin with \texttt{-}. any other list token will be ignored.
\item The only list token replaced in ordered lists is \texttt{\textbackslash{}d+.} (digits followed by a point: \texttt{1.}). If an other list token is used (a letter...), the numbered list won't be processed. If additional tokens are added (for nested lists: \texttt{1. a.}), the list will be processed, but the second token (\texttt{a.}) will not be deleted. This is because the use of additional tokens is specific to the Markdown interpreter used.
\end{itemize}
\item Titles and different title levels
\item Images
\item URLs, both in text and as Markdown hyperlinks.
\item Inline quotes : french and english style, nested quotes.
\item Footnotes. \textbf{Warning}: loose footnotes (references in the body that point to nothing or footnotes that point to nothing in the body) will be deleted. 
\end{itemize}
\subsection{Currently unsupported}

\begin{itemize}
\item Markdown tables
\item Strikethrough
\item Emojis
\item Tasks lists
\item Definition lists
\item Text blocks inside lists will be processed as the continuation of a list item 
\end{itemize}

\par\noindent\rule{\linewidth}{0.4pt}
\section{Things that will mess with the output}

The things in the list below will either cause the script to exit or produce a messy output.

\begin{itemize}
\item Poorly formed Markdown. If a list is not properly formed, the script will exit.
\item Using things that are currently not supported.
\item Incoherent use of header levels (\texttt{\#}, \texttt{\#\#}...) will be processed and render a valid document, but it will mess with the appearence of the TeX file and of the table of contents, especially if the headers are numbered.
\item Unnumbered lists made using non-standard markdown syntax (lists not beginning with \texttt{-}...). These will simply not be matched.
\item Numbered lists using additional numbering keys, such as \texttt{1. a}. These will be matched, but in this case, \texttt{1.} will be deleted and replaced by TeX syntax; \texttt{a} will be kept, so the TeX file will contain double numbering (the TeX numbering and this second numbering key)
\item Highly nested lists are not supported by default in LaTeX, but this can be changed in your preamble.
\item Multiple footnotes with the same key; for the footnote processing to work, there must be unique footnote numbers and only one note per footnote number.
\item Loose footnotes that point to nothing will be deleted. 
\end{itemize}

Clean your file first ! You can also make the bulk of the translation using this cli and fine tune your \texttt{.tex} 
file later on.

\par\noindent\rule{\linewidth}{0.4pt}
\section{About nested lists and visual indentation}

List nesting and visual indentation is a bit of a pickle, so here's an explanation of how
the indentation is processed and what constitutes a valid list for this tool.
\subsection{What is a valid list?}

All items must be \textbf{as indented than the first list item} or more. the indentation level of the first item
will be removed before processing that list. This means that \textit{the first example} will result in the same
result as \textit{the second}.

\begin{lstlisting}
- list item with no leading space
- same here
\end{lstlisting}

\begin{lstlisting}
  - list item with two leading spaces
  - same here, still a valid list
\end{lstlisting}

All indentation levels must be a multiple of the leading whitespaces of the first nested item. If the first nested 
list item is indented with 4 whitespaces, all nested items must be indented with a multiple of 4 spaces. For example,
a list item can't be indented with 3 spaces while another one is indented with 4 spaces.

The above two rules cause the script to exit with an error message pointing to the faulty list. On top of that,
in keeping with LaTeX and markdown logic, a list item can't skip a nesting level (you cannot go from an item in a list
to an item in a list in a list). This will \textbf{not} stop the script, the indentation will just be reset automatically.

\begin{lstlisting}
- this is an item at level 0
  - this is an item at level 1; the indentation is set as 2 whitespaces
  - same here; level 1.
   - visually, this one seems indented as level 3 as it starts with 6 leading whitespaces.
   however, its nesting level will be reset from level 3 to level 2 to avoid skipping
   a nesting level
  - this item is at level 1
\end{lstlisting}
\subsection{Technically, how does it work ?}

\begin{itemize}
\item The complete markdown list is matched.
\item Each list item is mapped to its indentation level, in number of leading spaces.
\item The list is validated. If it is not valid, the script stops.
\item An indentation multiplier is defined: it is the number of spaces in the first nested list item.
\item This multiplier is used to map each list item to its nesting level.
\item The markdown is syntax is replaced by TeX syntax. 
\end{itemize}
\subsection{Examples of valid and invalid lists}
\subsubsection{Valid lists}

\begin{lstlisting}
- item at level 0
- item at level 0
  - item at level 1. here, a 2-whitespace indentation is defined for the list.
   - item at level 2, with 4 leading spaces.
- item at level 0
\end{lstlisting}

\begin{lstlisting}
   - item at level 0
   - item at level 0
\end{lstlisting}

\begin{lstlisting}
- item at level 0
   - item at level 1. for this list, the indentation is set to 4 spaces.
   - visually, this item is at level 3 (12 leading spaces), but will be reindented to 2 spaces.
   - item at level 1.
\end{lstlisting}
\subsubsection{Invalid lists}

\begin{lstlisting}
  - item at level 0
  - item at level 0
- this will cause the script to exit: the item is less indented than the first item.
\end{lstlisting}

\begin{lstlisting}
- item at level 0
	- indentation is set at 4 spaces
	 - here, there are 6 spaces. the indentation is incoherent and the script stops.
\end{lstlisting}

\par\noindent\rule{\linewidth}{0.4pt}
\section{License and credits}

Developped by Paul Kervegan in August 2022 and released under GNU GPL v3.


\clearpage
\tableofcontents
\end{document}
